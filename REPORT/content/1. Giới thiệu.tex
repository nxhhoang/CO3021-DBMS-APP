\section{GIỚI THIỆU}
Trong kỷ nguyên công nghệ thông tin hiện đại, dữ liệu đóng vai trò cốt lõi trong việc vận hành và ra quyết định của mọi hệ thống phần mềm. Sự bùng nổ của Big Data và nhu cầu xử lý đa dạng các loại dữ liệu đã dẫn đến sự chuyển dịch từ việc thống trị tuyệt đối của các hệ quản trị cơ sở dữ liệu quan hệ sang sự ra đời và phát triển mạnh mẽ của các hệ quản trị cơ sở dữ liệu phi quan hệ (NoSQL). \medskip

Tuy nhiên, không có một giải pháp ``vạn năng'' cho mọi bài toán. Trong khi RDBMS (đại diện là \textbf{PostgreSQL}) nổi bật với tính nhất quán dữ liệu và tuân thủ chặt chẽ các thuộc tính ACID, thì NoSQL (đại diện là \textbf{MongoDB}) lại mang đến sự linh hoạt về lược đồ (Schema-less), khả năng mở rộng  và tốc độ truy xuất vượt trội đối với dữ liệu phi cấu trúc.\medskip

Trong phạm vi bài tập lớn này, nhóm tập trung nghiên cứu, so sánh và đánh giá hiệu năng giữa hai đại diện tiêu biểu: \textbf{PostgreSQL} (SQL) và \textbf{MongoDB} (NoSQL). Sự so sánh được thực hiện dựa trên 5 khía cạnh kỹ thuật cốt lõi của một hệ quản trị cơ sở dữ liệu:
\begin{itemize}[label=$\circ$]
    \item Data Storage \& Management (Lưu trữ và quản lý dữ liệu)
    \item Indexing (Chỉ mục)
    \item Query Processing (Xử lý truy vấn)
    \item Transaction (Giao dịch)
    \item Concurrency Control (Điều khiển đồng thời)
\end{itemize}

Để minh họa một cách thực tiễn sự khác biệt và khả năng bổ trợ lẫn nhau của hai hệ quản trị này, nhóm quyết định xây dựng một ứng dụng \textbf{Sàn Thương mại điện tử (E-commerce Platform)} với kiến trúc lai (Hybrid Architecture). Đây là miền ứng dụng đặc thù đòi hỏi sự kết hợp ưu điểm của cả hai loại CSDL:
\begin{itemize}[label=$\circ$]
    \item \textbf{PostgreSQL:} Chịu trách nhiệm quản lý các dữ liệu yêu cầu tính toàn vẹn cao, cấu trúc chặt chẽ và ràng buộc quan hệ phức tạp như: Thông tin người dùng (Users), Đơn hàng (Orders), Kho hàng (Inventory) và Thanh toán.
    \item \textbf{MongoDB:} Chịu trách nhiệm lưu trữ các dữ liệu có cấu trúc linh động, phân cấp hoặc khối lượng lớn như: Thông tin chi tiết sản phẩm (Products - với các thuộc tính đa dạng tùy loại hàng), Đánh giá (Reviews) và Log hệ thống.
\end{itemize}

Thông qua việc phân tích lý thuyết và thực nghiệm trên ứng dụng E-commerce, báo cáo này sẽ cung cấp cái nhìn sâu sắc về ưu nhược điểm của từng loại DBMS, từ đó đưa ra kết luận về ngữ cảnh sử dụng phù hợp nhằm tối ưu hóa hiệu năng hệ thống.
