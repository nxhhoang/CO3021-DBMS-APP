\clearpage
\section{KIẾN TRÚC ỨNG DỤNG}

\subsection{Mô hình tổng quát}
Hệ thống Sàn Thương mại điện tử được xây dựng dựa trên mô hình kiến trúc \textbf{Client-Server (3-tier architecture)}. Việc áp dụng mô hình này giúp tách biệt rõ ràng giữa giao diện người dùng (Presentation Layer), xử lý logic nghiệp vụ (Business Logic Layer) và lưu trữ dữ liệu (Data Access Layer), tạo điều kiện thuận lợi cho việc bảo trì và mở rộng hệ thống trong tương lai.

\begin{figure}[H]
    \centering
    \includegraphics[width=0.75\linewidth]{images/kienTrucUngDung.png}
    \caption{Sơ đồ kiến trúc tổng quan của hệ thống E-commerce}
    \label{fig:architecture}
\end{figure}

\subsection{Các công nghệ sử dụng}
Dựa trên các tiêu chuẩn phát triển web hiện đại và yêu cầu của đề tài, nhóm đề xuất bộ công nghệ (Tech stack) như sau:
\begin{itemize}[label=$\circ$]
    \item \textbf{Frontend:} Sử dụng thư viện \textbf{ReactJS} kết hợp với ngôn ngữ \textbf{TypeScript} để đảm bảo tính chặt chẽ, dễ kiểm soát lỗi của mã nguồn. Giao diện được thiết kế sử dụng \textbf{TailwindCSS} giúp tăng tốc độ phát triển và tối ưu trải nghiệm người dùng.
    \item \textbf{Backend:} Hệ thống Backend được xây dựng trên nền tảng \textbf{NodeJS} với framework \textbf{ExpressJS}. Đây là lựa chọn phù hợp cho kiến trúc hướng sự kiện (event-driven), xử lý tốt các tác vụ I/O non-blocking, đặc biệt hiệu quả cho các ứng dụng thương mại điện tử có lượng truy cập đồng thời lớn.
    \item \textbf{Database:} Áp dụng kiến trúc lai (Hybrid) với sự kết hợp giữa \textbf{PostgreSQL} và \textbf{MongoDB}.
\end{itemize}

\subsection{Thiết kế Cơ sở dữ liệu lai (Hybrid Database Design)}
Để tận dụng tối đa ưu điểm của cả hai dòng cơ sở dữ liệu SQL và NoSQL, nhóm áp dụng chiến lược \textbf{Polyglot Persistence} với sự phân chia trách nhiệm dữ liệu cụ thể như sau:

\subsubsection{PostgreSQL (RDBMS - SQL)}
PostgreSQL đóng vai trò là kho lưu trữ chính cho các dữ liệu quan trọng (Core Data), yêu cầu tính cấu trúc cao, toàn vẹn tham chiếu và tuân thủ chặt chẽ các thuộc tính ACID.
\begin{itemize}[label=$\circ$]
    \item \textbf{Users \& Authentication:} Lưu trữ thông tin tài khoản, mật khẩu và phân quyền (Role-based access control).
    \item \textbf{Orders \& Transactions:} Quản lý đơn hàng, chi tiết đơn hàng và lịch sử thanh toán. Đây là dữ liệu không được phép sai sót hay mất mát.
    \item \textbf{Inventory:} Quản lý tồn kho. Cơ chế Transaction và Locking của PostgreSQL giúp giải quyết triệt để bài toán cạnh tranh (Race condition) khi nhiều người dùng cùng đặt mua một sản phẩm cuối cùng.
\end{itemize}

\subsubsection{MongoDB (NoSQL - Document Store)}
MongoDB được sử dụng để lưu trữ các dữ liệu có cấu trúc linh động (Schema-less), yêu cầu tốc độ đọc ghi nhanh và khả năng mở rộng chiều ngang (Scaling out).
\begin{itemize}[label=$\circ$]
    \item \textbf{Products (Catalog):} Do tính chất đa dạng của các mặt hàng (ví dụ: Laptop có cấu hình CPU/RAM, Quần áo có Size/Màu sắc), việc sử dụng cấu trúc Document (JSON/BSON) cho phép lưu trữ linh hoạt mà không cần chuẩn hóa dữ liệu phức tạp như RDBMS.
    \item \textbf{Reviews \& Comments:} Dữ liệu đánh giá thường có khối lượng lớn và phi cấu trúc. MongoDB cho phép truy xuất nhanh nội dung này để hiển thị mà không cần thực hiện các phép JOIN tốn kém.
    \item \textbf{Logs:} Lưu trữ lịch sử hoạt động, hành vi người dùng (User behavior) để phục vụ phân tích dữ liệu và gợi ý sản phẩm (Recommendation).
\end{itemize}

\subsection{Phân tích và mô tả yêu cầu dữ liệu}
\subsubsection{Mô tả nền tảng}
Hệ thống E-commerce là một nền tảng thương mại điện tử hiện đại được xây dựng theo kiến trúc kết hợp giữa cơ sở dữ liệu quan hệ PostgreSQL và cơ sở dữ liệu NoSQL MongoDB (Polyglot Persistence). \medskip

Hệ thống phục vụ hai nhóm người dùng chính: khách hàng (customer) và quản trị viên (admin). \medskip

Khách hàng có thể đăng ký tài khoản, quản lý thông tin cá nhân và nhiều địa chỉ giao hàng, tìm kiếm sản phẩm theo nhiều tiêu chí, xem chi tiết sản phẩm, thêm sản phẩm vào giỏ hàng, thực hiện đặt hàng và theo dõi trạng thái đơn hàng. Ngoài ra, khách hàng có thể đánh giá và nhận xét các sản phẩm đã mua.\medskip

Quản trị viên có thể quản lý danh mục sản phẩm, thêm, chỉnh sửa hoặc xóa sản phẩm, theo dõi tồn kho, quản lý đơn hàng và thống kê doanh thu.\medskip

Hệ thống đảm bảo:
\begin{itemize}[label=$\circ$]
    \item Tính toàn vẹn dữ liệu giao dịch thông qua PostgreSQL
    \item Tính linh hoạt và mở rộng dữ liệu sản phẩm thông qua MongoDB
    \item Khả năng xử lý đồng thời và bảo mật đăng nhập thông qua cơ chế token
\end{itemize}

Các chức năng chính bao gồm: quản lý người dùng, quản lý sản phẩm, tìm kiếm nâng cao, xử lý đơn hàng, thanh toán, quản lý tồn kho, đánh giá sản phẩm, ghi log hành vi người dùng và thống kê hệ thống.

\subsubsection{Mô tả các kiểu thực thể, thuộc tính và mối liên kết}
Cơ sở dữ liệu của hệ thống E-commerce quản lý thông tin về người dùng, vai trò, sản phẩm, đơn hàng, kho hàng, thanh toán, đánh giá và các hoạt động liên quan.

\begin{itemize}[label=$\circ$]
    \item Thực thể \texttt{User} (Người dùng - PostgreSQL): Mỗi người dùng có các thuộc tính: \texttt{userId} (định danh duy nhất), \texttt{email} (duy nhất), \texttt{password}, \texttt{fullName}, \texttt{phoneNum}, \texttt{createdAt}, \texttt{role}. Mỗi người dùng có thể có nhiều địa chỉ giao hàng. Mỗi người dùng có thể tạo nhiều đơn hàng. Mỗi người dùng có thể có nhiều token đăng nhập (\texttt{refresh token}) để quản lý phiên làm việc.
    
    \item Thực thể \texttt{Address} (Địa chỉ người dùng - PostgreSQL). Mỗi địa chỉ gồm: \texttt{addressID} duy nhất, \texttt{addressLine}, \texttt{city}, \texttt{district}, \texttt{isDefault}. Mỗi người dùng có thể có nhiều địa chỉ giao hàng.
    
    \item Thực thể \texttt{AuthToken} (Phiên đăng nhập - PostgreSQL). Mỗi bản token gồm: \texttt{tokenID} duy nhất, \texttt{refreshToken}, \texttt{userAgent}, \texttt{expiresAt}, \texttt{isRevoked}. Mỗi \texttt{token} thuộc về đúng một người dùng, cho phép quản lý đăng nhập từ nhiều thiết bị. 
    
    \item Thực thể \texttt{Inventory} (Kho hàng). Mỗi bản ghi của kho gồm: \texttt{inventoryID} duy nhất, \texttt{productID} (tham chiếu logic tới MongoDB), \texttt{sku} duy nhất, \texttt{stockQuantity}, \texttt{lastUpdated}. Mỗi sản phẩm có thể có nhiều biến thể kho (SKU khác nhau).

    \item Thực thể \texttt{Order} (Đơn hàng - PostgreSQL). Mỗi đơn hàng có: \texttt{orderID} duy nhất, \texttt{status} (\texttt{PENDING}, \texttt{PROCESSING}, \texttt{SHIPPED}, \texttt{DELIVERED}, \texttt{CANCELLED}), \texttt{totalAmount}, \texttt{shippingAdd} (snapshot), \texttt{createdAt}. Mỗi đơn hàng thuộc về một người dùng. Mỗi đơn hàng gồm nhiều chi tiết đơn hàng (order item). Mỗi đơn hàng có đúng một thanh toán (\texttt{payment}).
    
    \item Thực thể \texttt{Item} (Chi tiết đơn hàng - PostgreSQL). Mỗi hàng hoá trong đơn hàng bao gồm: \texttt{itemID} (duy nhất), \texttt{productID}, \texttt{productName} (snapshot), \texttt{quantity}, \texttt{unitPrice}. Mỗi \texttt{OrderItem} thuộc về đúng một đơn hàng.

    \item Thực thể \texttt{Payment} (Thanh toán - PostgreSQL). Mỗi thực thể bao gồm: 
    \texttt{paymentID} (duy nhất), \texttt{method}, \texttt{status}, \texttt{transactionDate}. 
    Mỗi thanh toán gắn với đúng một đơn hàng.

    \item Thực thể \texttt{Review} (Đánh giá – MongoDB). Mỗi đánh giá bao gồm:
    \texttt{reviewID} (duy nhất), \texttt{productID}, \texttt{userID}, \texttt{userName}, \texttt{rating} (1--5 sao), \texttt{comment}, \texttt{images}, \texttt{createdAt}. 
    Một người dùng có thể gửi nhiều đánh giá và một sản phẩm có thể nhận nhiều đánh giá từ nhiều người dùng.

    \item Thực thể \texttt{UserActivityLog} (Log hành vi – MongoDB). Mỗi bản log bao gồm: 
    \texttt{logID} (duy nhất), \texttt{userID} (có thể null), \texttt{actionType}, \texttt{targetID}, \texttt{metadata}, \texttt{timestamp}. 
    Hệ thống ghi nhận các hoạt động như xem sản phẩm, tìm kiếm, thêm vào giỏ hàng.

    \item Thực thể \texttt{Product} (Hàng hoá - \texttt{MongoDB}). Mỗi sản phẩm có: \texttt{productID} duy nhất, \texttt{name}, \texttt{slug}, \texttt{category}, \texttt{basePrice}, \texttt{description}, \texttt{images}, \texttt{attributes} (thuộc tính động như \texttt{RAM}, màu sắc, kích thước, cấu hình...), \texttt{isActive}, \texttt{createdAt}. Một sản phẩm có thể nhận nhiều đánh giá (review). Sản phẩm không lưu trực tiếp số lượng tồn kho mà được tham chiếu logic sang \texttt{PostgreSQL}.
\end{itemize}

\subsubsection{Các ràng buộc ngữ nghĩa không biểu diễn được bằng (E-)ERD}

\begin{itemize}[label=$\circ$]
    \item Email người dùng phải duy nhất và hợp lệ.
    \item Số lượng tồn kho không được âm.
    \item Tổng tiền đơn hàng phải lớn hơn hoặc bằng 0.
    \item Một đơn hàng chỉ có một bản ghi thanh toán.
    \item Trạng thái đơn hàng phải tuân theo luồng xử lý hợp lệ (Pending → Processing → Shipped → Delivered hoặc Cancelled).
    \item Refresh token chỉ hợp lệ trong thời gian cho phép và có thể bị thu hồi.
    \item Một sản phẩm chỉ được phép đặt hàng khi còn tồn kho.
    \item Đánh giá sản phẩm chỉ được gửi sau khi người dùng đã hoàn thành đơn hàng chứa sản phẩm đó.
    \item Các thuộc tính động của sản phẩm phải phù hợp với danh mục sản phẩm.
    \item Khi một sản phẩm bị xóa, các bản ghi kho và đơn hàng liên quan phải được xử lý tương ứng.
    \item Dữ liệu liên kết giữa PostgreSQL và MongoDB phải được đồng bộ chính xác thông qua productID và userID.
\end{itemize}

\clearpage
\subsubsection{Các biểu đồ mô tả}
\paragraph{Biểu đồ EERD cho SQL database}\text{ }
\begin{figure}[H]
    \centering
    \includegraphics[width=0.8\linewidth]{images/EERD.png}
    \caption{Biểu đồ EERD cho SQL database}
    \label{fig:EERD_PostgreSQL}
\end{figure}

\paragraph{Biểu đồ mô tả cho NoSQL database}\text{ }
\begin{figure}[H]
    \centering
    \includegraphics[width=\linewidth]{images/mongoDB_EERD.png}
    \caption{Biểu đồ mô tả cho NoSQL database}
    \label{fig:EERD_mongoDB}
\end{figure}

\paragraph{Biểu đồ mô tả các quan hệ logic giữa SQL và NoSQL}\text{ }
\begin{figure}[H]
    \centering
    \includegraphics[width=0.6\linewidth]{images/logicRelationSQLAndNoSQL.png}
    \caption{Biểu đồ mô tả các quan hệ logic giữa NoSQL database và SQL database}
    \label{fig:EERD_mongoDB}
\end{figure}

\subsection{Ánh xạ các lược đồ Cơ sở dữ liệu}
\subsubsection{Ánh xạ mô hình dữ liệu của SQL}
\begin{figure}[H]
    \centering
    \includegraphics[width=0.8\linewidth]{images/mapping.png}
    \caption{Lược đồ Cơ sở dữ liệu}
    \label{fig:EERD_mongoDB}
\end{figure}

\subsubsection{Ánh xạ bộ sưu tập dữ liệu của NoSQL}
\begin{itemize}[label=$\circ$]
    \item Sản phẩm (Products) - Ảnh (images):
    \begin{itemize}[label=\textbullet]
        \item Quan hệ: 1 - Ít (Mỗi sản phẩm chỉ khoảng 5-10 ảnh).
        \item Quyết định: Embed. Lưu trực tiếp mảng URL ảnh trong document sản phẩm để lấy ra ngay lập tức.
    \end{itemize}
    \item Sản phẩm (Products) - Thuộc tính (attributes):
    \begin{itemize}[label=\textbullet]
        \item Quan hệ: 1 - 1 (Mỗi sản phẩm có 1 bộ thông số).
        \item Quyết định: Embed. Dữ liệu này luôn đi kèm sản phẩm và thay đổi cấu trúc tùy loại hàng (Polymorphic Pattern).
    \end{itemize}
    \item Sản phẩm - Đánh giá (reviews):
    \begin{itemize}[label=\textbullet]
        \item Quan hệ: 1 - Rất nhiều (Một sản phẩm có thể có hàng nghìn review).
        \item Quyết định: Reference. Nếu lồng review vào sản phẩm sẽ dễ bị tràn giới hạn 16MB của MongoDB. Do đó, tách Review ra collection riêng và tham chiếu bằng product\_id.
    \end{itemize}
\end{itemize}
\begin{figure}[H]
    \centering
    \includegraphics[width=0.6\linewidth]{images/NoSQLCollections.drawio.png}
    \caption{Bộ sưu tập dữ liệu}
    \label{fig:EERD_mongoDB}
\end{figure}

















